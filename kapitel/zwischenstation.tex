
\chapter{Die Zwischenstation}

Als Zwischenstation lässt sich eine Station bezeichnen, bei der ich aus der Intensivstation entlassen und in eine normales Krankenhauszimmer gesteckt wurde. Täglich wurde mir die Medikamente verabreicht und von einer Krankenschwester vorbeigebracht, aber ansonsten wurde ich in Ruhe gelassen.
Vor der Zwischenstation wurde der Katheter entfernt, ein Vorgang der sehr schmerzhaft war. 

In der erster Übernachtung entdeckte ich, dass durch den Schlaganfall inkontinent geworden bin. Da ich zu dem Zeitpunkt noch erleuchtet war, war ich darüber nicht sonstig aufgeregt. Allerdings war der Situation etwas peinlich. Eine Krankenschwester war mein Bett machen und entdeckte die Bescherung. Wie stark die Inkontinenz war wusste ich noch nicht. Im Badezimmer gab es ein paar Unterhosen, die hilfreich waren. 

Zum Frühstück gab es zwei Brötchen, etwas Butter und Marmelade. Und eine Tüte mit einem Pulver, die mir nicht verständlich war. Laut Pfleger war die Tüte wichtig, so trank ich sie mit etwas warmen Wasser. 

Danach hatte ich Zeit. Auf einem Zettel waren Übungsaufgaben eines Logopäden um die durch den Schlaganfall beschädigten Hirnpartien wieder auf den Vordermann zu bringen. Da ich noch auf Besuch eines Arztes wartet, fing in der Zwischenzeit an die Zettel ausfüllen. Erst beim ausfüllen, wurden mir klar, wie groß mein Sprachlücken waren. Die Lücken schienen nur die Sprache zu betreffen, meine Gedanken funktionierten noch. Gemäß dem Zettel musste ich die Inkontinenz. Dabei viel mir auf, wie dass ich das Wort Inkontinenz nicht schreiben konnte.  Einen Helfer der zufällig kam, bat ich um Hilfe beim Schreiben der ersten Wörter, um sie mit der Visite besprechen zu können. Zum ersten Mal wurde wurde mir bewusst, wie viel ich verloren hatte. Ich begann jämmerlich zu weinen.



Ich könne aber in der Zeit spazieren gehen. Mir wurde gesagt wie viel Zeit ich hätte bis zur Visite, so ging ich erst kleinere Runden, dann immer große. Irgendwie war mir klar, ich war auf eine Station für Erleuchtete bzw. der Erleuchtung Verdächtige. Ich beobachte die Leute auf meiner Station. Einige schienen zu sterben. Sie waren alt und eingefallen. 

Aber dazwischen ziehen war auch normale. Neben meinem Zimmer wohnte eine Frau, ca. vierzig oder fünfzig. Ich fragte Sie wann es Mittagessen gäbe, aber sie reagierte abweisend und ohne Stimme. Sie sah aus, als wenn sie in ein Diktiergerät gesprochen hätte. Vermutlich hat sie versucht, wie ich, ihre Sprache wieder herzustellen, aber war bislang erfolglos. Sie schlug mit Empörung ihr Tür zu. 


Einer sah aus, wie Jesus. Aber die Augen sahen leer aus. Er bekam regelmäßig Besuch von seiner Familie
, Frau, Sohn und vermutlich Vater. Die Familie reagierte nicht auf meine Blicke und lief traurig dem Mann davon. Mein hätte gern dem Mann geholfen, vielleicht hätte ihm


