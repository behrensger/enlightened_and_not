
\chapter{Die Zwischenstation}

Als eine Zwischenstation lässt sich diejenige Station bezeichnen, auf die ich von der Intensivstation verlegt und in ein normales Krankenhauszimmer gelegt wurde. Täglich wurden mir die Medikamente verabreicht und von einer Krankenschwester gebracht, aber ansonsten wurde ich in Ruhe gelassen.
Vor der Verlegung auf die Zwischenstation wurde der Katheter entfernt, ein Vorgang, der sehr schmerzhaft war.

In der ersten Übernachtung entdeckte ich, dass ich durch den Schlaganfall inkontinent geworden war. Da ich zu dem Zeitpunkt noch erleuchtet war, war ich darüber nicht sonderlich aufgeregt. Allerdings war die Situation etwas peinlich. Eine Krankenschwester machte mein Bett und entdeckte das Malheur. Wie stark die Inkontinenz war, wusste ich noch nicht. Im Badezimmer gab es ein paar Einwegslips, die hilfreich waren.

Zum Frühstück gab es zwei Brötchen, etwas Butter und Marmelade. Und eine Tüte mit einem Pulver, das mir nicht verständlich war. Laut dem Pfleger war die Tüte wichtig, also trank ich es mit etwas warmem Wasser.

Danach hatte ich Zeit. Auf einem Zettel waren Übungsaufgaben eines Logopäden, um die durch den Schlaganfall geschädigten Hirnpartien wieder auf Vordermann zu bringen. Da ich noch auf den Besuch eines Arztes wartete, fing ich in der Zwischenzeit an, die Zettel auszufüllen. Erst beim Ausfüllen wurde mir klar, wie groß meine Sprachlücken waren. Die Lücken schienen nur die Sprache zu betreffen, meine Gedanken funktionierten noch. Auf dem Zettel musste ich das Wort "Inkontinenz" schreiben. Dabei fiel mir auf, dass ich das Wort "Inkontinenz" nicht schreiben konnte. Einen Pfleger, der zufällig vorbeikam, bat ich um Hilfe beim Schreiben der ersten Wörter, um sie bei der Visite besprechen zu können. Zum ersten Mal wurde mir bewusst, wie viel ich verloren hatte. Ich begann jämmerlich zu weinen. Gleichzeitig war ich aber auch glücklich. Ich war ja erleuchtet und das weinen hörte auf.

Ich begann, die Welt vor der Zimmertür zu erkunden. Neben meinem Zimmer lag eine Frau, ungefähr vierzig oder fünfzig. Ich fragte sie, wann es Mittagessen gäbe, aber sie reagierte abweisend und stumm. Sie sah aus, als ob sie in ein Diktiergerät gesprochen hätte. Vermutlich hatte sie versucht, wie ich, ihre Sprache wiederherzustellen, aber war bislang erfolglos. Sie schlug empört ihre Tür zu. Ich war zuerst traurig wegen dieser Rücksichtslosigkeit. Aber ziemlich schnell wurde ich mitfühlend. Die Frau hatte das gleiche durchgemacht wie ich und verdiente etwas Mitgefühl.

Andere Frauen, sie sahen teilweise wie Krankenschwestern aus, schienen zum Mittagessen zu gehen. Aus Vorsicht, ich wusste ja nicht wie die Krankenschwestern reagieren, ging ich einen Seiteneingang hinaus. Nach einem kurzem Ausflug und Rückweg, wurde ich mich sicher und begann mich im Krankenhaus umzusehen. 

Das Krankenhaus kam mir verwirrend vor. Ich musste mich Stückweise vortasten, bis mich zurechtfand. Auf meinem Herumschleichen kam ich an einem Zimmer mit dem Foto einer Nonne vorbei. Das Türschild versprach Führung bzw. Seelsorge. Führung brauchte ich dringend, allerdings war die Tür verschlossen. Fast kamen mir wieder die Tränen, ich schluckte sie runter und lief den Gang weiter. Dabei traf ich kurz einen Mann. Der Mann schien es eilig zu haben und zielorientiert den Gang entlang zumarschieren. Der Mann spielt später noch eine Rolle.

Den Gang weiter, kam ein Krankenhaus-Kirche bzw. -Kapelle.  In der Kapelle saßen drei Nonnen. Aber das Entscheidende war mein Gefühlsleben. ich hatte mir Führung gewünscht und es kam Jesus Christus. Mir kamen die Tränen. Ein der Nonnen wollte wissen was mit mir ist. Ich erzählte was möchte, dass ich auf meine Frau und meinen Sohn warte, dass ich aber vergessen hätte wie sie heißen, aber dass sie ganz gewiss kämen. Die Nonne antworte darauf irgendetwas, aber sie sprach zu leise und ich konnte sie nicht verstehen. Aber mein sehnlichster Wunsch, nach spiritueller Begleitung wurde erfüllt. Ich wünschte der Nonne einen schönen Tag und ging weiter.

Vor der Tür kam mir der Mann von vorhin entgegen. Jetzt wurde er mir langsam auffällig. Was suchte der Mann?

Auf weiterer Wanderschaft kam ich an den Eingangsbereich. Auffällig war der verwinkelte Grundriss des Krankenhauses. Der Mann, der mir jetzt schon zweimal aufgefallen war, viel mir auch hier wieder auf. Ich traf ihn erneut und allmählich dämmerte mir, was die Ursache seinen Verhaltens war. Er geistig verwirrt und wollte entweder raus oder rein aus dem Krankenhaus, aber fand den Weg nicht. Der schwierige Grundriss machte jetzt Sinn. 

Ich könne aber in der Zeit spazieren gehen. Mir wurde gesagt wie viel Zeit ich hätte bis zur Visite, so ging ich erst kleinere Runden, dann immer große. Irgendwie war mir klar, ich war auf eine Station für Erleuchtete bzw. der Erleuchtung Verdächtige. Ich beobachte die Leute auf meiner Station. Einige schienen zu sterben. Sie waren alt und eingefallen. 

Einer sah aus, wie Jesus. Aber die Augen sahen leer aus. Er bekam regelmäßig Besuch von seiner Familie
, Frau, Sohn und vermutlich Vater. Die Familie reagierte nicht auf meine Blicke und lief traurig dem Mann davon. Mein hätte gern dem Mann geholfen, vielleicht hätte ihm


